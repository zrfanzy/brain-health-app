
%%% Local Variables:
%%% mode: latex
%%% TeX-master: t
%%% End:

\chapter{引言}
\label{cha:intro}

\section{研究背景}

认知功能障碍指的是记忆障碍和轻度的其他认知功能障碍,是痴呆的早期症状,重要的临床意义在于早期发现和早期干预,可以延迟甚至阻止痴呆的发展。\footnote{摘自中华老年医学杂志2006年7月第25卷第7期 《中国防治认知功能障碍专拣共识》}然而由于认知功能障碍表现形式非常多样化,在不同个体上体现的如记忆力快速下降、时间能力受损、计算能力障碍、空间定向障碍、语言理解执行能力下降、判断力下降、逻辑判断能力障碍等等,诊断需要对被试人各个方面(记忆,逻辑,计算,语言,绘画等等)均进行测试,故传统的诊断方式为使用医疗问卷进行医患一对一的评估。医疗问卷涵盖了上述各个方面的测试,测试内容包括故事复述题、重复数字题、逻辑选择题、记忆绘画题、回忆绘画题等等,问卷长达103页,医生使用手册长达131页,要学习如何正确打分需要至少一年的训练和学习。在测试过程中医生除了需要在纸上勾选、记录病人的回答,也需要用音频记录病人的一些回答,有一些题目也需要病人在纸上进行绘画,测试中还需要医生通过一些现场观察进行一些问卷之外的诊断,这些纸质和媒体记录的方式虽然为诊断提供了大量的信息,也使得数据量不断庞大,这些基于不同媒体介质的数据使得后期病例整理非常麻烦。

信息技术的发展使得移动健康变成了医药产业的一个重要的分支,医疗服务信息化也已经成为了国际趋势,越来越多的国内外医院开始引进信息技术来辅助一些医疗服务\footnote{引自“Design and implementation of doctor-patient interaction system based on android”}。北京协和医院和波士顿大学与清华大学寻求合作,希望共同开发脑健康评估系统,目的在能基于安卓系统开发一套医患交互系统,能够利用移动互联网的优势辅助认知功能障碍的诊断,并能在平板电脑上完全还原传统纸板的系统,能够系统地组织整理测试所得数据(文字答案,语音和绘画轨迹),同时也界面设计也希望能够方便易懂、使医生能够在段时间内学会如何使用系统。

\section{研究现状}

美国、日本以及一些欧洲国家非常重视相关疾病的研究,美国建立了近30个专业研究中心,每年的研究经费超过10亿美元,欧洲国家在这个领域上投入也超过1亿欧元。经过十多年的临床实验和数据分析,国际上有一些公认的简易诊断工具,如Mini-Cog,MMSE,MoCA,GPCOG等等,这些工具需要测试患者的认知功能的各个方案,但这些工具只能提供辅助性的判断,如需确诊还需要进一步更深入的测试。波士顿大学向我们提供的《神经心理学测试长问卷》\footnote{原名:“Neuropsychological Test Battery”}就是这样一个“更深入”的测试问卷,这个长达103页的测试包括多种题型,全面测试患者的记忆、逻辑、计算、语言、绘画等认知功能的多方面功能,通过这些测试结果综合评分,可以确定测试患者是否已患认知功能障碍。

虽然该疾病领域的研究已经发展了十多年,但仍然停留在传统的纸质长问卷上,目前还没有一个自动诊断系统,“目前所有应用于医学测试的系统都只是文字版的”\footnote{引自“Systems and methods for the physiological assessment of brain health and the remote quality control of EGG systems”}。

\section{设计内容}

本文提出了一个基于android系统、建于平板电脑的脑健康评估系统。本系统为配合医生进行认知性功能障碍诊断使用,设计并实现了多种题型包括选择题、轨迹记录题、故事复述题、单词对记忆题、数字串复述题、图片展示题等展示和使用的界面,并包含了数据记录、音频录制、轨迹记录等多个功能。为后期数据方便整理,对题目格式、数据格式针对系统进行了统一的规划和设计,并对数据的存储位置进行了考量和设计。在界面中加入了防错和引导功能,使医生能够在段时间迅速学会使用这个系统进行诊断。

\section{论文的组织结构}

本文第一张详细描述了本系统设计和开发的背景,主要介绍了国内为认知功能障碍的诊断方法现状,讨论了设计和开发本系统的重要意义。

第二章介绍了本次开发使用的平台、工具和问卷资料。

第三章根据问卷资料和医院的需求进行了需求分析,并根据需求对整个系统和数据存储进行了框架设计。

第四章根据系统需要的语音识别功能进行了相关实验并提出了使用百度云语音识别的结论。

第五章根据框架设计,针对每一个模块进行了细节分析、设计和实现。

第六章介绍了对系统的测试。

第七章对本系统设计的实现进行了总结。

