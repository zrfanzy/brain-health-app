
%%% Local Variables: 
%%% mode: latex
%%% TeX-master: t
%%% End: 

\chapter{总结}

本文主要基于认知功能障碍传统的长问卷制作了辅助诊断的一整个系统,实现了画图题、录音题、选择题等多种题型。该系统建立在Android操作系统上,具有记录轨迹、录制和播放音频、逻辑判分等功能,也拥有独立和统一的数据系统。该系统也具有比较好的复用性和拓展性。

通过本次毕业设计,基于一个拥有多题型的问卷,根据其需求设计了一个能够方便自定义、结构和数据统一化的系统,并按照设计实现了三个版本的APP。在这个过程中,经历了一系列相关领域、功能的调研,试用了一些现有的服务、进行了实验来验证性能并选择了使用的服务;然后经过参考多个系统的结构,设计了一整套可用、统一的系统,并对自己的设计进行了实现,最后在选择的硬件上安装APP进行了测试,并得到了合作方的确认。

在本次毕业设计中,我的主要贡献在提出了基于三个版本(在线医生版、在线病人版和离线打分版)、数据和结构统一化的系统设计,其将各功能分开的特点具有编码简单、能够自定义各类题目以及扩展性和复用性强的好处;规定了题目和答案的存储方案,具有扩展性强、自由性强、安全性强、找回率高的特点;另外也根据题目的需求设计了各题的展示界面,不仅复原了纸版问卷的题目,同时具有方便易用、也易学会的特点。